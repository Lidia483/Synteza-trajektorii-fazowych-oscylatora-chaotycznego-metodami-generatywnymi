W swojej książce David Blockley zauważył: „Niepodważalne jest, że inżynieria stanowi serce społeczeństwa"~\cite{david2012}. Inżynieria zajmuje się zarówno aspektami teoretycznymi, jak i praktycznymi, co umożliwiaja rozwój innowacyjnych rozwiązań. Celem niniejszej pracy inżynierskiej jest połączenie tych dwóch elementów poprzez implementację estymatora Theil-Sen na mikrokontrolerze STM32.

 Metoda Theil-Sen jest jedną z dokładniejszych technik stosowanych w analizie regresji, cieszącą się szczególnym uznaniem wśród metod odpornych na wartości odstające. Jedną z jej głównych zalet jest wysoka odporność na punkty odstające w zbiorach danych, co czyni ją niezawodnym narzędziem analitycznym nawet przy nieprecyzyjnych wynikach pomiarów. Umożliwia oszacowanie nachylenia prostej oraz punktu przecięcia z osią Y, dzięki czemu znajduje zastosowanie w szerokim zakresie dziedzin – od ekonomii, przez inżynierię, po metrologię.

Estymator został po raz pierwszy opisany przez Henriego Theila w artykule z 1950 roku. Theil zauważył, że dotychczasowe metody wyznaczania regionów ufności zakładały normalność rozkładu błędów. W swojej pracy pisał:
„Regiony ufności dotychczas wyznaczano, zakładając normalność rozkładu. Głównym celem tego artykułu jest pokazanie, jak je wyznaczać bez takiego założenia” ~\cite{theil1950}.
Jego podejście pozwoliło na przeprowadzanie analizy bez konieczności przyjmowania, że błędy pomiarowe spełniają założenia o rozkładzie normalnym (rozkładzie Gaussa), co stanowiło przełom w ówczesnej statystyce.

W 1968 roku Pranab Kumar Sen, drugi z autorów tej metody, rozszerzył jej możliwości, uwzględniając przypadki, w których dwa punkty ze zbioru danych mają tę samą współrzędną x. W swojej pracy Sen pisał:
„Niniejsze badanie skupia się na opracowaniu wielowymiarowego podejścia do tworzenia testów porządkowych opartych na takich samych rangach” ~\cite{sen1968}.
Dzięki temu metoda jest w stanie radzić sobie z bardziej złożonymi przypadkami, co znacznie zwiększyło jej uniwersalność i zastosowanie w analizie danych o powtarzających się wartościach.

Porównując metodę Theil-Sen z klasyczną metodą najmniejszych kwadratów, można zauważyć istotne różnice. Metoda najmniejszych kwadratów, mimo swojej popularności i szerokiemu stosowaniu, nie radzi sobie najlepiej, gdy ma miejsce zniekształcenie wyników przez wartości odstające. Metoda Theil-Sen, dzięki swojej odporności na nieprawidłowości w danych, często okazuje się bardziej wiarygodnym wyborem w przypadku, gdy wyniki mimo anomalii w wartościach muszą być precyzyjne i niezawodne.

Ze względu na swoje zalety i uniwersalność, metoda Theil-Sen znajduje zastosowanie w wielu środowiskach programistycznych oraz na różnych platformach sprzętowych. Szczególnie ważna jest jej implementacja na mikrokontrolerach, które często służą do analizy danych w systemach czasu rzeczywistego. Celem niniejszej pracy dyplomowej jest implementacja metody Theil-Sen na mikrokontrolerze STM32, z wykorzystaniem języka C oraz środowiska programistycznego STM32CubeIDE. Proces implementacji będzie obejmował opracowanie algorytmu, wyznaczanie współczynnika nachylenia i punktu przecięcia z osią y oraz optymalizację obliczeń pod kątem ograniczonych zasobów obliczeniowych mikrokontrolera.