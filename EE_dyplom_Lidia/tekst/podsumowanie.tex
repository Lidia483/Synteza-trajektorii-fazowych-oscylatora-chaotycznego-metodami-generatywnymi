Niniejsza praca dotyczy implementacji estymatora Theil-Sen na mikrokontrolerze STM32, będącego metodą obliczania linii regresji odporną na wartości odstające. Kluczową zaletą tej metody jest jej niezawodność w przypadku danych pomiarowych, które mogą być zakłócone lub zniekształcone przez szum, co jest częstym zjawiskiem w pomiarach opartych na czujnikach. Dodatkowo estymator Theil-Sen został porównany z klasyczną metodą najmniejszych kwadratów, aby uwypuklić jego wady i zalety oraz zidentyfikować odpowiednie obszary jego zastosowań.
 

Przegląd literatury wskazuje, że metoda Theila-Sena cieszy się dużym zainteresowaniem w różnych dziedzinach analizy danych. W prognozowaniu cen Bitcoina, gdzie występuje wysoka zmienność, wykazuje ona większą odporność na wartości odstające w porównaniu do innych metod, co czyni ją bardziej efektywną w warunkach szybko zmieniającego się rynku~\cite{bitcoin2021}. Z kolei w artykule z 2024 roku estymator Theil-Sen został zastosowany w analizie danych hydrologicznych i atmosferycznych, gdzie wykazał lepszą skuteczność niż metoda najmniejszych kwadratów~\cite{zahra2024}. Szczególnie wyróżniał się odpornością na wartości odstające, zapewniając bardziej wiarygodne wyniki. Dodatkowo charakteryzował się mniejszym błędem obliczeniowym w porównaniu do innych metod regresyjnych.

 
Przeprowadzone w pracy testy potwierdzają, że estymator Theil-Sen jest lepszym wyborem do zadań, gdzie występują skrajne odchylenia w danych, podczas gdy klasyczna metoda najmniejszych kwadratów może prowadzić do zniekształconych wyników, co jest zgodne z obecnym stanem wiedzy. Krytyczna analiza wyników wskazuje, że wybór metody zależy od charakterystyki danych wejściowych oraz wymaganej odporności na błędy pomiarowe.

W pracy zaimplementowano estymator Theil-Sen w języku C z użyciem środowiska STM32CubeIDE oraz zaprojektowano układ pomiarowy z płytką PCB, co pozwoliło na przeprowadzenie testów w realnych warunkach. Testy te wykazały, że mimo ograniczonych zasobów mikrokontrolera STM32, możliwe jest skuteczne realizowanie zaawansowanych algorytmów statystycznych. Wartością dodaną projektu jest wykazanie, że nawet systemy wbudowane, o ograniczonej pamięci i mocy obliczeniowej, mogą być używane do realizacji zaawansowanych algorytmów, co wpisuje się w aktualne kierunki rozwoju technologii.

Podsumowując, celem pracy było opracowanie i implementacja algorytmu Theil-Sen w środowisku STM32CubeIDE, co zostało osiągnięte. Implementacja została przeprowadzona poprawnie, algorytm działa zgodnie z założeniami, a wyniki testów potwierdzają jego efektywność. Ponadto, praca wykazała użyteczność estymatora Theil-Sen w systemach wbudowanych, szczególnie w zastosowaniach wymagających analizy danych z płytek PCB. Mimo pewnych ograniczeń związanych z pamięcią i mocą obliczeniową udało się uzyskać wyniki o wysokiej jakości, co dowodzi, że metoda ta jest użytecznym narzędziem w takich systemach.
Cel pracy został osiągnięty, a założenia dotyczące porównania metod, implementacji oraz analizy wyników zostały zrealizowane. W przyszłości warto rozważyć dalszą optymalizację algorytmu oraz rozszerzenie implementacji o dodatkowe metody statystyczne. 